%% 
%% ACS project dissertation template. 
%% 
%% Currently designed for printing two-sided, but if you prefer to 
%% print single-sided just remove ",twoside,openright" from the 
%% \documentclass[] line below. 
%%
%%
%%   SMH, May 2010. 


\documentclass[a4paper,12pt,twoside,openright]{report}


%%
%% EDIT THE BELOW TO CUSTOMIZE
%%

\def\authorname{From Context Embeddings to Meaning Embeddings \xspace}
\def\authorcollege{David Yenicelik \\ ETH Zürich\xspace}
\def\authoremail{yedavid@ethz.ch}
\def\dissertationtitle{}
\def\wordcount{14,235}


%\usepackage[dvips]{epsfig,graphics} 
\usepackage{epsfig,graphicx,verbatim,parskip,tabularx,setspace,xspace}
\usepackage{amsmath, amsfonts}
\usepackage{bbm}

\usepackage[ruled,vlined]{algorithm2e}
\usepackage{tabularx,booktabs}


%% START OF DOCUMENT
\begin{document}


%% FRONTMATTER (TITLE PAGE, DECLARATION, ABSTRACT, ETC) 
\pagestyle{empty}
\singlespacing
% title page information
\begin{titlepage} 

\begin{center}
\noindent
\huge
\dissertationtitle \\
\vspace*{\stretch{1}}
\end{center}

\begin{center}
\noindent
\huge
\authorname \\
\Large
\authorcollege      \\[24pt]
%\begin{figure}
\includegraphics{uni.jpg}
%\end{figure}
\end{center}

\vspace{24pt} 

\begin{center}
\noindent
\large
{\it A dissertation submitted to ETH Zürich \\ 
in partial fulfilment of the requirements for the degree of \\ 
Master of Science ETH in Computer Science} 
\vspace*{\stretch{1}}
\end{center}

\begin{center}
\noindent
Eidgenössische Technische Hochschule Zürich \\
Data Analytics Lab     \\
Institute of Machine Learning  \\
CAB F 42.1, Universitätsstrasse 6, 8006, Zürich\\
Zürich 8006 \\
{\sc Switzerland}    \\
\end{center}

\begin{center}
\noindent
Email: \authoremail \\
\end{center}

\begin{center}
\noindent
\today
\end{center}

\end{titlepage} 

\newpage
\vspace*{\fill}

\onehalfspacing
\input{declaration}
\singlespacing
\newpage
{\Huge \bf Abstract}
\vspace{24pt} 

One of the goals of Natural Language Understanding \textit{NLU} is to formalize written text in a way, such that linguistic features can be captured through computational models, which can then again be used for downstream machine learning tasks.
The most popular methodology in NLU is the use of vectors that represent written text, including word-vectors, sentence-vectors, etc. 
These vectors are often referred to as \textit{embedding} vectors, as they embed more complex concepts into a vector representation.

Because embeddings - due to their strong usage in downstream tasks - can be considered as the fundamental unit of machine learning in the domain of natural language, any shortcomings will propagate to the performance of the target task.
Thus, improving the way these embeddings have strong implications for any subtasks in NLU.

Given that language models such as BERT only capture a function black-box function $f$ between input and output, our aim is to understand how the the subspace organization relates to linguistic features.
Due to the strong importance of \textit{semantics}, which captures meaning in text, we will focus our analysis on \textit{semantics}.

Our main contribution are two-fold. 
We first test to what extent the semantic subspace is linearly separable between semantic classes, and then test for multi-modality in these structures through a clusterability test.
We then analyse the relationship between the linguistic features of part-of-speech and semantics, as these show strong correlation in languages.
Finally, we introduce new embedding-vectors for tokens with high variance, and try to understand what downstream tasks are affected the most, using this as a proxy to understand what the strong word-based variance in BERT context vectors most corresponds to.

Our findings show that although BERT does not capture an interpretable concept of a semantic subspace, but that it rather focuses on the more broad linguistic features which are imminent to context.
This can often introduce undesirable features such as strong sentiment, position in sentence, and thus lead to considerable bias in downstream applications.

%- Some properties include that static word vectors like word2vec form a bijection between discrete vectors and word tokens. 
%However, because a single word can entail multiple meanings, such as the polysemous word "bank" ( (1) financial institution, (2) a sitting bench ), this results in a lossy compression
%- Other language models like context embeddings entail too much information, and also include other linguistic features such as semantic information, relatenedness to unrelated concepts.
%These properties can easily introduce bias into any downstream tasks.
%- We conjecture that many language tasks incl. translation will benefit most from meaning information
%
%- Our general approach is to start with a complex language model that outputs context embeddings, and find signals / vectors that entail meaning.
%- Although this work is dedicated to the domain of natural language understanding, the principles analysed in this work should generalize to other domains with similar structural properties as well, where we want to denoise some embedding space to some select properties.

\newpage
\vspace*{\fill}


\pagenumbering{roman}
\setcounter{page}{0}
\pagestyle{plain}
\tableofcontents
\listoffigures
\listoftables

\onehalfspacing

% Definition of macros
\newcommand{\norm}[1]{\left\lVert#1\right\rVert}
\newcommand{\bracket}[1]{\left[#1\right]}
\newcommand{\absdet}[1]{\left|#1\right|}
%% START OF MAIN TEXT 

%\chapter{Introduction}
\chapter{Introduction}

%TODO Re-do this part
\pagenumbering{arabic} 
\setcounter{page}{1} 

\newpage
\chapter{Motivation}
 
- Natural Language Understanding (NLU) lies in the intersection of formalising human text into a computer-readable format.
- Computer-readble units must be numerical, thus we represent words and meanings by vectors
- The relationships between vectors should cover underlying relations between the word meaning
- Language models are a generalization of word vectors
- Word embeddings such as Word2Vec and language models such as BERT are used for other tasks, which are refered to as "downstream" tasks, due to their nature of using up these word-embeddings and language models

- Because these are the most basic units of text, any shortcomings and properties will propagate over to any downstream task
- Some properties include that static word vectors like word2vec form a bijection between discrete vectors and word tokens. 
However, because a single word can entail multiple meanings, such as the polysemous word "bank" ( (1) financial instution, (2) a sitting bench ), this results in a lossy compression
- Other language models like context embeddings entail too much information, and also include other linguistic features such as semantic information, relatenedness to unrelated concepts.
These properties can easily introduce bias into any downstream tasks.
- We conjecture that many language tasks incl. translation will benefit most from meaning information

- Our general approach is to start with a complex language model that outputs context embeddings, and find signals / vectors that entail meaning.
- Although this work is dedicated to the domain of natural language understanding, the principles analysed in this work should generalize to other domains with similar structural properties as well, where we want to denoise some embedding space to some select properties.


%\chapter{Background} 
\chapter{Background} 

TODO: Take some more from here
%TODO http://da.inf.ethz.ch/teaching/2019/CIL/lecture/CIL2019-05-Word-Embeddings.pdf

- Some of the main points behind \cite{harris54} are that there is inherent structure in language, and that this structure can be formalized.
- \cite{harris54} mentions that the relation between the linguistic representation in terms of sounds and tokens are related to the meaning that the representation entail.
- However, despite the obvious relationship, the distinction between distributional structure of the language and meaning is not always clear. and that there is a "parallel" meaning structure, and argues that there is not a one-to-one relation between vocabulary and classification of meaning. 
- Also argues that the meaning of a word is entailed by it's environment, and the words that it occurs with.
- Other viewpoints (CITE HOFMANNs "teachers") In the end, language captures the evolution of human thought.

- cite paper "meaning is classified by its context"

Chronologically, the following data structures are manifestations of the above idea of defining a word by it's neighbourhood.

\section{Word Embeddings}

\paragraph{Word-Embeddings}
In general, we want to find a mapping 

\begin{equation}
w^t \mapsto (x_w, b_w) \in \mathcal{X}^{d + 1}
\end{equation}{\label{map:embedding_mapping}}

where $w^t$ is a token representation from a vocabulary $w^t \in V$ and where this token representation is transformed into a $d$-dimensional vector representation $x_w$, and a bias-term $b_w$ that is specific to the token $w^t$.
Whenever we will talk about \textit{word-vectors}, \textit{(word)-embeddings}, or \textit{feature-vectors}, we will refer to the image of the above map. 
For convenience and unless stated otherwise, we will assume that $x_w$ absorbs the bias term $b_w$ as an additional vector-element.
Also, for simplicity and unless otherwise stated, we will use the euclidean real-valued vector-space $mathbb{R}^{d+1}$ to described the resulting embedding vectors.
Please note, however, that the choice of the embedding space $\mathcal{X}$ is not fixed in general, and as such, work in other spaces have also been conducted. 
%TODO Cite Gerry


\paragraph{Distance} Now our goal is to build an embedding space where the relationship between words are meaningful.
Specifically, we want to incorporate the notion of \textit{meaning} into these word-embeddings, which should follow following properties.
Formally, we introduce the concept of \textit{distance} $d : \mathcal{X}  \times \mathcal{X} \mapsto [ 0, \inf )$, which should capture the relation between different elements. 
For a set of elements $x, y, z \in \mathcal{X}$, following properties must hold for our distance metric to be a valid on:

\begin{enumerate}
\item $d(x, y) \geq 0$ (non-negativity)
\item $d(x, y) = 0 \iff x = y$ (identity, i.e. if two embedding vectors are identitical, they must capture the same underlying word instance)
\item $d(x, y) \leq d(x, z) + d(z, y)$ (triangle inequality)
\end{enumerate}{\label{def:distance}}

One consequence of the above rules is that for words $a, b, c$, each having a word embedding $x, y, z$ respectively, $d(x, y) < d(x, z) \iff $ word instance $b$ is conceptually closer to word $a$ than word $c$.
For convenience, whenever I input $a, b, c$ into the distance function $d$, the word-embeddings for the corresponding word-tokens shall be used.
Also, please notice that I left out the notion of symmetry for distance measures.

\paragraph{Learning a distance}
Often in machine learning, one wants to maximize a certain probability distribution given some data $\mathbf{X}$.
In the context of word vectors, we want to maximize the probability that $w$ occurs in the context window of $w \prime$ through some parameters $\theta$.
Implicitly, this corresponds to minimizing the distance between $w$ and $w^{\prime}$, while keeping the distance between $w$ and all other words constant. 
We call $w \prime$ a \textit{context word} for $w$.

\begin{equation}
p \left(w | w^{\prime}\right)
\end{equation}

and 

\begin{equation}
\forall w, w^{\prime} : \hspace{20pt} \text{max } p \left(w | w^{\prime}\right) \iff \text{min } d(w | w^{\prime})
\end{equation}

\paragraph{Going from the distributional structure of sentences to learning distances between words.} One of the early formalizations of representing the distributional structure of words through was expressed in \cite{bengio03}, which argues that a sequential statistical model can be constructed to estimate this true posterior.  
In it's most naive form, this would imply that we can estimate the probability of a word $w^{(t)}$ after words $w^{(t-1)}, \ldots, w^{(1)}$ as

\begin{align}
p(\mathbf{w}) &= p(w^{(t)}, \ldots, w^{(1)}) \\
&= p\left( w^{(t)} | w^{(t -1)}, \ldots, w^{(1)} \right)
\end{align}{\label{eq:naive_sequential_probability}}

%TODO check the underlying equation again!!

where $\mathbf{w} = w^{(1)}, \dots, w^{(T)} $ is the sequence of words, $p(\mathbf{w})$ the probability of this sentence occurring.

Because inference for the above equation would have been computationally infeasible, the notion of \textit{n-grams} was introduced to estimate this conditional probability, following a markov assumption.

\begin{align}
p\left( w^{(t)} | w^{(t -1)}, \ldots, w^{(1)} \right) \approx p\left( w^{(t)} | w^{(t -1)}, \ldots, w^{(t - n + 1)} \right)
\end{align}

Often, the above equation will be simplified even further using the independence of words assumption

However, because we are only interested in the distance between two words at a time (and not the full seqence) \cite{bengio03} simplify the above function using the assumption of marginalized independence of words amongst each other w.r.t. a target word $(t)$ (we can do this because we marginalize over all possible combinations of target words and context words).

\begin{equation}
p(\mathbf{w})=\prod_{t=1}^{T} \prod_{\triangle < t} p\left( w^{(t)} | w^{(t -\Delta)}\right)
\end{equation}{\label{eq:naive_sequential_probability}}


 whose probability we wish to estimate (and maximize if this is in the given training dataset), $\mathcal{I}=\{-R, \ldots,-1,1, \ldots, R\}$ is the so-called \textit{context window} which includes all the words that $R$ index units to the left, and $R$ index units to the right of the word of interest $w^(t)$.

\paragraph{However, we do not need to only look at only the previous words}.
One can consider the full neighbourhood of a word.
If we describe this by a loss-measure we would like to minimize, this would result in 

%TODO I guess this is the skip-gram approach, the bag-of-words is again a different one

\begin{equation}
\mathcal{L}(\theta ; \mathbf{w})=\prod_{t=1}^{T} \prod_{\triangle \in \mathcal{I}} p_{\theta}\left(w^{(t)} | w^{(t +\Delta)}\right)
\end{equation}{\label{eq:basic_equation_log_maximization}}

where $\mathbf{w} = w^{(1)}, \dots, w^{(T)} $ is the sequence of words whose probability we wish to estimate (and maximize if this is in the given training dataset), $\mathcal{I}=\{-R, \ldots,-1,1, \ldots, R\}$ is the so-called \textit{context window} which includes all the words that $R$ index units to the left, and $R$ index units to the right of the word of interest $w^(t)$.

%TODO rephrase the structuring of the n-grams. I don't like how it is introduced here

If $p_\theta$ is a parametric function, one can then optimize for the most optimal $\hat{\theta} = \text{argmax}_\theta \mathcal{L}(\theta ; \mathbf{w})$ using a maximum likelihood estimation approach.

In general, one does not necessarily need to interpret $\mathbf{w}$ as a sequence of word-tokens, but can also interpret $w^{(1)}, \ldots, w^{(T)}$ as \textit{n-grams}, which are triplets of n-characters forming a token-unit.
Generally, the definition of a token is open for interpretation. 
We will generally assume that a single word is a token unless otherwise stated.
%TODO Cite some Token Paper!!

Intuitively the above models fllw the idea expressed by \cite{harris54} very well, speaking that the meaning of a word is captured by its neighborhood.
However, the above equations follow a \textit{continuous bag of words } (CBOW) approach.
A continuous-bag-of-words approach is an approach where we want to predict a target word $w^t$ given some context words $w^{\prime}$.
However, it is also possible to follow a \textit{skip-gram} approach.
Here, we want to predict context words $w^{\prime}$ given a target word $w^t$.

\begin{figure}[h]
	\center
  \includegraphics[width=0.6\linewidth]{./assets/background/cbow_and_skipgram.png}
  \caption{Figure taken from \cite{mikolov13}. The CBOW architecture predicts the current word based on the context. The Skip-gram predicts surrounding words given the current word.}
  \label{fig:cbow_skipgram}
\end{figure}

Using a skip-gram approach, \eqref{eq:basic_equation_log_maximization} for example would be reformulated into 
\begin{equation}
\mathcal{L}(\theta ; \mathbf{w})=\prod_{t=1}^{T} \prod_{\triangle \in \mathcal{I}} p_{\theta}\left(w^{(t +\Delta)} | w^{(t)}\right)
\end{equation}{\label{eq:basic_equation_log_maximization_skipgram}}


The question now is, how do we parametrize the distribution $p_\theta(w, w^{\prime})$, as well as the word feature vectors $x_w$ and $b_w$. 
The following will show a few methods of how this can be achieved.
In the following methods shown, we will aim to provide a loss function that we try to minimize, and interesting properties for each method.
However, we will not go into too much details, as to how the loss function is optimized, as almost all of these methods can be solved using gradient-based methods.

\newpage
\subsection{Static Word Embeddings}

Here we will talk about word-embeddings where each word-token only has a single $x_w$. 
Specifically, the mapping \eqref{map:embedding_mapping} is not a probabilistic function with an implicit random factor, but rather a deterministic one.

\subsubsection{Basic Model}

The first model we are going to look at is a most basic model which fulfills the properties of the distance metrics shown in \eqref{def:distance}.

Here, we can introduce a \textit{log-bilinear} model where the log-probability is define as

\begin{equation}
\text{log} p_{\theta}(w | w^{\prime}) = \left\langle\mathbf{x}_{w}, \mathbf{x}_{w^{\prime}}\right\rangle+b_{w} + \text { const. }
\end{equation}

To arrive at the actual probability, we can exponentiate the log-probability as such

$$
p_{\theta}\left(w | w^{\prime}\right)=\frac{\exp \left[\left\langle\mathbf{x}_{w}, \mathbf{x}_{w^{\prime}}\right\rangle+b_{w}\right]}{Z_{\theta}\left(w^{\prime}\right)}
$$

where $Z_{\theta}\left(w^{\prime}\right):=\sum_{v \in \mathcal{V}} \exp \left[\left\langle\mathbf{x}_{v}, \mathbf{x}_{w^{\prime}}\right\rangle+b_{v}\right]$ is a normalization constant such that the probability mass sums to $1$, and the model parameters entail the word-embeddings $\theta = (x_w, b_w) \in \mathbb{R}^{d+1}$

%TODO CIL lecture slides
% http://da.inf.ethz.ch/teaching/2019/CIL/lecture/CIL2019-05-Word-Embeddings.pdf


\subsubsection{Word2Vec}

%TODO look at this more closely again

During training, the above basic model comes with certain drawbacks. The distance would be minimized if all word-embeddings would collapse onto a single point. 
Also, there is no term that forces unrelated words to move away from each other, a property that we are interested in as unrelated words should form embedding vectors that are far from each other.

One of the most prominent example of word vectors manifests itself in the work of \cite{mikolov13} and \cite{mikolov13b}.
Here, a neural network with a single embedding layer can be trained to transform one-hot-vectors $\in \{ 0, 1 \}^{| \mathcal{V} |}$ which represents a word $w$ in vocabulary $\mathcal{V}$
into a latent vector representation $w \in \mathbf{R}^{d + 1}$ using both a continuous bag of word and also a continuous skip-gram approach.
The skip-gram approach is preferred in practice.

Specifically, the loss-function that is optimized looks as follows

\begin{align} 
\mathcal{L}(\theta ; \mathbf{w}) & =\sum_{t=1}^{T} \sum_{\Delta \in \mathcal{I}} [\\ 
& b_{w^{t+\Delta}} +\left\langle \mathbf{x}_{w^{(t+\Delta)}}, \mathbf{x}_{w^{(t)}} \right\rangle \\
& -\log \sum_{v \in \mathcal{V}} \exp \left[\left\langle\mathbf{x}_{v}, \mathbf{x}_{w^{(t)}}\right\rangle+b_{v} \right] 
\end{align}

As one can see, the loss function takes in the bilinear loss from the basic model, and complements this by adding a term, such that random samples are not put next to each other.
This principle is often referred to as \textit{negative sampling}.

\begin{figure}[h]
	\center
  \includegraphics[width=0.6\linewidth]{./assets/background/word2vec_cities.png}
  \caption{Figure taken from \cite{mikolov13b}. A 2-dimensional PCA projection of the 1000-dimensional skip-gram vectors of countries and their capital cities. The proposed model is able to automatically organize concepts and learn implicit relationships between them. No supervised information was provided about what a capital city means.}
  \label{fig:cbow_skipgram}
\end{figure}

%TODO include difference between target vocabulary V and context vocabulary C

\subsubsection{GloVe}

For the global vectors for word representation (GloVe) \cite{pennington14}, the authors follow a more straight-forward matrix factorization approach.

First, a global co-occurence matrix is created $\mathbf{N} = (n_{ij}) \in \mathbb{N}^{|\mathcal{V}| \dot |\mathcal{C}|}$ where each entry $n_{ij}$ is determined by the number of occurrences of word $w_i \in \mathcal{V}$ in context $w_j \in \mathcal{C}$.
Given that the vocabulary size can exceed multiple thousand items, this practically results in a sparse matrix.

\begin{align}
\mathcal{H}(\theta ; \mathbf{N}) &=
\sum_{i, j} f\left(n_{i j}\right)(\underbrace{\log n_{i j}}_{\text {target }}-\underbrace{\log \tilde{p}_{\theta}\left(w_{i} | w_{j}\right)}_{\text {model }})^{2} \\
&= \sum_{i, j} f\left(n_{i j}\right)(\log n_{i j} - \left\langle x_i, y_j \right\rangle )^{2} \\
\end{align}

where $f(n_{ij})$ is the weighting function which assigns a weight for each entry in the co-occurence matrix. 
In the second line, we also again use a bilinear probability density model $\tilde{p}_{\theta}\left(w_{i} | w_{j}\right)=\exp \left[\left\langle\mathbf{x}_{i}, \mathbf{y}_{j}\right\rangle+b_{i}+c_{j}\right]$.
The constants $b_i, c_j$ are left out and are assumed to be absorbed in the embedding vectors.

A popular choice for the weighting function is 

$$
f(n) = \min \left\lbrace 1, \left(\frac{n}{n_{\max}}\right)^{\alpha} \right\rbrace
$$

with $\alpha \in (0, 1]$.
The motivation behind this is that frequent words do not receive a weighting which is "too high" (there is a cutoff at some point) and small counts are considered noise and slowly cancelled out.


\subsubsection{Gaussian Embeddings}

Gaussian Embeddings have been first proposed in the context of words, although prior work has been adapted in embedding matrix-rows into a mean and standard deviation \cite{vilnis14} 

It is a continuous probabilistic relaxation of the otherwise common discrete point vectors.
Each word is represented by a Gaussian distribution in high-dimensional space, with the aim to better capture uncertainty and a representation and it's relationships.
It can also express asymmetric relations more naturally than dot-products or cosine similarity, and enables for better-parametrized rule between decision boundaries.

Training is done using an energy function which is incorporated within a loss function that we try to minimize. 
The energy function describes similarity between two items.
The authors propose the following energy functions to derive a Gaussian word-embedding.

\begin{equation}
L_m(w, c_p, c_n) = max(0, m - E_\theta(w, c_p) + E_\theta(w, c_n)
\end{equation}

Here, $w$ is the word we want to sample, $c_p$ is a "positive" context word, i.e. a word that co-occurs with the word $w$, and $c_n$ is a "negative" context word, i.e. a word that does not co-occur with the word $w$.
Usually the negative context words is sampled randomly from the corpus.
The loss function reminds of a hinge-loss in logistic regression.

The authors propose two possible ways to learn the mean and variance of the Gaussian embeddings.
They argue that the empirical covariance is not the most effective method of deriving the words as Gaussian embeddings.
This does not allow for inclusion between ellipsoids 

\paragraph{Symmetric similarity: expected likelihood or probability product kernel}

We can use any kernel (which is symmetric by definition) to derive at an energy function.
For two Gaussians $f(x)$, $g(x)$, the inner product is defined as:

\begin{align}
E_\theta(w, c) &= \int_{x \in \mathcal{R}^d} f(x)g(x) dx \\
&= \int_{x \in \mathcal{R}^d} \mathcal{N}(x; \mu_w, \Sigma_w) \mathcal{N}(x; \mu_c, \Sigma_c) dx \\
&= \mathcal{N}(0; \mu_w - \mu_c, \Sigma_w + \Sigma_c)
\end{align}

For numerical feasibility and easy of differentiation, we usually maximize the $\text{log} E_\theta(w, c)$ for a given dataset with $w \in \mathcal{W}, c \in \mathcal{C}$.

\paragraph{Asymmetric divergence: KL-Divergence}

We can use more directional supervision to exploit directional supervision, such as a knowledge graph.

Following energy-function is optimized:

\begin{align}
-E(w_i, c_j) & = D_{KL}(c_j || w_i) \\
&= \int_{x \in \mathcal{R}^d} \mathcal{N}(x; \mu_{w_i}, \Sigma_{w_i}) \text{log} \frac{\mathcal{N}(x; \mu_{c_j}, \Sigma_{c_j})}{\mathcal{N}(x; \mu_{w_i}, \Sigma_{w_i})} dx \\
&= \frac{1}{2}\left(\operatorname{tr}\left(\Sigma_{i}^{-1} \Sigma_{j}\right)+\left(\mu_{i}-\mu_{j}\right)^{\top} \Sigma_{i}^{-1}\left(\mu_{i}-\mu_{j}\right)-d-\log \frac{\operatorname{det}\left(\Sigma_{j}\right)}{\operatorname{det}\left(\Sigma_{i}\right)}\right)
\end{align}

Because of the loss function, this can entail information such as "y entails x" as a soft form of inclusion between two datasets (if KL divergence is used).
If a symmetric loss function is used, then this would most likely lead to overlap (IS THIS TRUE...???)

%TODO Veerify this statement again

\paragraph{Uncertainty calculation:} In contrast to the empirical standard deviation as an uncertainty measure, we can now calculate the uncertainty of the inner product (i.e. the distribution $P(z=x^T y)$ using the following formula

\begin{align}
\mu_z = \mu_x^T \mu_y
\Sigma_z = \mu_{x}^T \Sigma_{x} \mu_{x}+\mu_{y}^T \Sigma_{y} \mu_{y}+\operatorname{tr}\left(\Sigma_{x} \Sigma_{y}\right)
\end{align}

We then get an uncertainty bound, where $c$ denotes the number of standard deviations away from the mean.
\begin{equation}
\mu_{x}^{\top} \mu_{y} \pm c \sqrt{\mu_{x}^{\top} \Sigma_{x} \mu_{x}+\mu_{y}^{\top} \Sigma_{y} \mu_{y}+\operatorname{tr}\left(\Sigma_{x} \Sigma_{y}\right)}
\end{equation}

We can learn the parameters $\Sigma$ and $\mu$ for each of these embeddings using a simple gradient-based approach, where we set hard constraints on 

\begin{align}
& \norm{ \mu_i }_2  \leq C, \forall i \\
& m I <  \Sigma_i < M I
\end{align}

The method shows competitive scores to the Skip-Gram model, although usually only with minor improvements depending on the benchmark-dataset.

%TODO Include one part about tokenizers

\newpage
\subsection{Context Embeddings}

Now that we have seen a few methods that aim to capture one embedding $x_w$ for each word-token, we not turn out attention to context embeddings.

Context embeddings are more modern applications of embeddings which 

A single transformer takes as input a sequence $x$ and outputs softmax probabilites.
In the context of sentences, it is able to take as input a full sequence of words.
This has implications for the complexity of the model.
Instead of calculating the raw probability of \eqref{naive_sequential_probability}, and even using a markovian assumption because computation power is expensive, the transformer architecture implicitly calculates the joint probability of an entire sequence of words withouth any explicit conditioning

%TODO: Rephrase this part

\begin{align}
p(w^{(t-d)}) &= p(w^{(t)}, \ldots, w^{(t-d + 1)}, w^{(t-d - 1)}, \ldots, w^{(1)})
\end{align}{\label{eq:transformer_probability}}

where $p(w^{(t-d)})$ is the probability of the target word $w^{(t-d)}$ which we want to calculate.

This modifies our initial ask of calculating $p(w, w^{\prime})$ to not including only a single context word $w^{\prime}$, but rather a wider context \\ (i.e. $w^{(t)}, \ldots, w^{(t-d + 1)}, w^{(t-d - 1)}, \ldots, w^{(1)}$).
Theoretically, and because we want to arrive at some distance measure, $p(w, w^{\prime})$ can still be calculated by marginalizing out all the context-variables except the one of interest.
Specifically, if we are interested in word $w$, we can marginalize out all input words except $w^{\prime}$.
Practically this is infeasible due to the high space, and as such, we make use of some other mechanisms that we will talk about in the "BERT" section below.

%TODO How to marginalize these over?
%TODO Talk about what kind of input these take-in
%TODO Look at bit more closely into the attention mechanism?
%TODO Can draw some more inspiration from this here

Because the following models not merely static word embeddings, but language models, we shall provide a short comparison between for downstream tasks between the individual language models after a short introduction on the details of each language models.

\subsubsection{ELMo}

The simplest idea of context embeddings, which take into account more than just a single word, but all tokens from a predefined context is the \textit{Embeddings from Language Models} (ELMo) model proposed in \cite{peters17}.

In its underlying architecture, ELMo is using LSTMs.
LSTMs are sequential recurrent neural networks .
%TODO how do you best explain LSTMs?

Specifically, LSTMs \cite{hochreiter97} are good at estimating the probability of word $w^{(t)}$ occurring given a previous history of words $w^{(t-1)}, \ldots w^{(1)}$.
In practice, the LSTM is a popular choice for sequence modeling tasks, as it is able to capture very well the following probability distribution.

\begin{equation}
p\left(w_{1},  w_{2}, \ldots, w_{N} \right)=\prod_{k=1}^{N} p\left(w_{k} | w_{k-1}, \ldots, w_{2}, w_{1}\right)
\end{equation}

The ELMo language model now makes use of a backward LSTM, which does not evaluate the probability of a certain token occuring with its \textit{past} words (i.e. the context with context tokens $w^{(k)}$ with $k < t$ for a word $w^{(t)}$ whose probability we want to estimate), but rather considers only the \textit{future context}.
Specifically, this results in 

%TODO Insert an example of what this could look like
\begin{equation}
p\left(w_{1},  w_{2}, \ldots, w_{N} \right)=\prod_{k=1}^{N} p\left(w_{k} | w_{k+1}, w_{k+2}, \ldots, w_{N}\right)
\end{equation}

A bidirectional language model (biLM) combines the above two LSTM models, thus jointly maximizing the log likelihood of the forwards, as well as backward directions.

\begin{align} 
\sum_{k=1}^{N} &\left(\log p\left(w_{k} | w_{k-1}, \ldots, w_{1} ; \Theta_{x}, \vec{\Theta}_{L S T M}, \Theta_{s}\right)\right.\\
+&\left.\log p\left(w_{k} | w_{k+1}, \ldots, w_{N}; \Theta_{x}, \overleftarrow{\Theta}_{L S T M}, \Theta_{s}\right)\right) 
\end{align}

Because the output of a sequential neural network - which the LSTM is part of - includes $2L + 1$ output representations, we receive the following set of vectors, one for each input token, which we now consider the \textit{context embeddings}.
Each context embeddings outputs a vector representing $x_{k}$ for the word token $w^{(k)}$, given its context $ w_{1}, \ldots, w_{k-1}  w_{k+1}, \ldots, w_{N}$.

\begin{align} 
R_{k} &=\left\{\mathbf{x}_{k}^{L M}, \overrightarrow{\mathbf{h}}_{k, j}^{L M}, \overleftarrow{\mathbf{h}}_{k, j}^{L M} | j=1, \ldots, L\right\} \\ &=\left\{\mathbf{h}_{k, j}^{L M} | j=0, \ldots, L\right\} 
\end{align}

Notice that one can stack the above biLM such that the output of one module is taken as input to another module.
The biLM module is stacked twice ELMo.
If a downstream task is trained end-to-end, the all output vectors can be stacked together and provided as input to a final downstrea-specific neural network module.

\subsubsection{The Transformer Architecture}

All of the below presented models use the transformer architecture which initially was presented in \cite{vaswani17}, which shows how far simple \textit{attention} modules can go.

%TODO Go into what attention is

\begin{figure}[h]
	\center
  \includegraphics[width=0.6\linewidth]{./assets/background/transformer_module.png}
  \caption{Figure taken from \cite{vaswani17}. The transformer module architecture. The transformer encapsulates multiple attention layers.}
  \label{fig:cbow_skipgram}
\end{figure}



The following architectures are all based on the transformer module.

\subsubsection{BERT}

As introduced in \cite{devlin18}, the Bidirectional Encoder Representations from Transformers model (BERT) combines the above concepts of forward and backwards sequential models with attention, as marked in the transformer architecture above.

\cite{devlin18} argue that the main limitations is that standard language models are unidirection, and that .

BERT introduces itself as a pre-trained language model.
Specifically, it shall be used to fine-tune on specific downstream tasks.

BERT is pre-trained on %TODO 
multiple gigabytes of text .

%TODO talk about masked training

%TODO Talk about tokenizer (wordpiece)

\begin{figure}[h]
	\center
  \includegraphics[width=0.6\linewidth]{./assets/background/BERT_multiple_input_tokens.png}
  \caption{Figure taken from \cite{devlin18}. BERT takes as input multiple tokens, including a position embedding, the token embedding and the segment embedding. This allows BERT to distinguish between the location of the word within a sentence, and which word token was provided and which sentence the word token is a part of.}
  \label{fig:cbow_skipgram}
\end{figure}


Multiple versions of BERT are provided, including $BERT_{LARGE}$ which includes %TODO input number of parameters
, and $BERT_{BASE}$ which includes %TODO number of parameters


\subsubsection{GPT and GPT-2}

First proposed in \cite{radford18} and further extended in \cite{radford19}.

%TODO read the two papers and take out anything significantly different to BERT.


\subsection{Other methods}

Although the above presented methods are the prevailent methods for word-embeddings, there are also other methods which do not clearly fit into one of the above categories.

\subsubsection{Generating "static" word-embeddings through contextual embeddings}

Some work has been done in extracting word-embeddings from contextual language models like BERT or ELMo.

CITE (BERT WEARS GLOVES: DISTILLING STATIC EMBEDDINGS FROM PRETRAINED CONTEXTUAL REPRESENTATIONS)

(1) Uses \textit{pooling} between BERT tokens to arrive at a single representation between words.

Here, sentences are split by space (tokenized).
Words are tokenized further into a subword as defined by WordPiece (Wu et al. 2016).

The defined pooling operations looks as follows to arrive at the word from the individual subwords:

$$
\mathbf{w}_{c}=f\left(\mathbf{w}_{c}^{1}, \ldots, \mathbf{w}_{c}^{k}\right) ; f \in\{\min , \max , \text { mean, last }\}
$$

where we have subwords $w^{1},  \ldots, w^{k}$ such that $\operatorname{cat}\left(w^{1}, \ldots, w^{k}\right)=w$

Why would any of these pooling operations result in a meaninigful source-word? 
This is just squishing tokens together! \\

-> This is a major limitation for which we may need to use ELMo
-> However this may be needed for "unseen concepts" (which are unseen words...)
-> Perhaps check what fasttext does...?


(2) Uses \textit{context combination} to map from different contexts $c_1, \ldots, c_n$ to a single static embedding $w$ that is agnostic of context.

Proposed are two ways to represent context.

\textbf{Decontextualization} For a single word-context, we siimply feed-in the word by itself to the model.

\textbf{Aggregated} combine $w$ in multiple contexts.
n sentences are sampled from the dictionary $\mathcal{D}$.
From the multiple sampled words, we then apply pooling to arrive at a single representation that aggregates the different tokens into one.

$$
\mathbf{w}=g\left(\mathbf{w}_{c_{1}}, \dots, \mathbf{w}_{c_{n}}\right) ; g \in\{\min , \max , \text { mean }\}
$$


This post extracts (token?) word-embeddings: 
(https://towardsdatascience.com/nlp-extract-contextualized-word-embeddings-from-bert-keras-tf-67ef29f60a7b)

This seems to be a way to extract embeddings for tokens from BERT
(https://github.com/imgarylai/bert-embedding)

(->How can we create a (parametric) probability density from a point-cloud distribution?)

perhaps not necessarily interpretable in standard euclidean space
(https://www.kdnuggets.com/2019/02/bert-features-interbertible.html)
original (https://medium.com/thelocalminima/are-bert-features-interbertible-250a91eb9dc)

Perhaps we can mask all but the target token to arrive at one vector per token (and then combine them somehow...).
But how do they extract the singular word-embeddings...?

(-> you could be like "acquiring bedeutung" is a big problem in many tasks. especially useful when we try to map one concept to another. we look at the NLP task for concreteness)

Generally, really good critique on this paper:

(https://openreview.net/forum?id=SJg3T2EFvr)

usually, we have sentence-embeddings, and do not look at word-embeddings.

(-> we don't want to add more and more context. we want a model which contains the polysemy of different contexts, which could allow for probability maximization..., otherwise we have to look at bigger and bigger documents to build more accurate language models, which becomes infeasible at some point. (although this would be the way humans work, because they live in context as well)

This blog aims to generate word-embeddings (and sentence-embeddings) from the BERT model.
(https://mccormickml.com/2019/05/14/BERT-word-embeddings-tutorial/)

create word-vectors by taking out what BERT predicts at the nth token.
create word-vectors by concatenating or summing multiple layer's outputs.

the cosine similarity between these vectors seem pretty well-done!


(-> Does it make sense to use BERT and then on calculate word-embeddings through an extended fully-connected model)

-> ELMo may provide a better tokenizer, maybe better ot use this? What about GPT? ELMo uses moses tokenizer which seems word-level enough

-> How to solve this tokenization problem....

-> Can also analyze only words that exist.


\section{Resources and Datasets}

\subsubsection{WordNet}

\subsubsection{SemCor dataset}

\subsubsection{News dataset}

\subsubsection{GLUE benchmark dataset}







\chapter{Related Work}

\section{Structure inside BERT}

(How Contextual are Contextualized Word Representations? Comparing the Geometry of BERT, ELMo, and GPT-2 Embeddings)

going down the drain of "geometry" of BERT and ELMo.

could also go down the drain of bias (we would prefer to have uniform space over gender etc.)

-> does projection into some subspace which has same metric properties perhaps not make it asitropic?

pretty ok summary of what kind of properties we want from word-embeddings... (https://devopedia.org/word-embedding)


Especially in Named Entity Recognition (NER), there is a lot of use for static word-embeddings.
I guess this is because we need static embeddings which represent the individual clusters?

-> Using pooling for some 

-> Character level operation

-> Perhaps make good sense to work towards a word-embeddings where different vectors are close to each other?

-> perhaps find a metric space warping the vectors, s.t. an isotropic representation is achieved?

-> Perhaps tokenization is a big problem, but perhaps other architecture..? but retraining is too difficult.. probably best to just stick to BERT? one way or the other, we need good word-embeddings derived from good language models to form a probabilistic prediction of the concept


-> Could perhaps also try to make an adversarial autoencoder after the BERT layer (or continue training, s.t. a second loss is mimized as a downstream task?)

-> Perhaps distilling with "correct" tokens? i.e. another network which copies BERT, but instead of outputting \#\#end, it outputs one of most frequent 20k words

-> thesaurus using a (set of) words. a little like sentence-generation, but generating most-probable examples

-> Everyone just averages token-embeddings..

-> perhaps fitting a GMM to the contextualized representations of BERT may give a good probability space..?

-> Perhaps make sense to apply MUSE to this?

-> Artetxe 2019 uses language models to generate embeddings. we also do this, but do it using 1) better language models, and 2) better 

-> QUESTION: Which factors (mapping algo, embedding) is delimiting in automated embedding matching

-> perhaps create a GMM for each concept, based on how many modals we identify? how to estimate the number of clusters? by graph-clustering perhaps! (this could be very consuming)

-> Adversarial autoencoder on BERTs last models to enforce it to some better distribution

\section{Metric Learning and Disentanglement}

\section{Zero shot and One shot learning }

\section{Clustering Algorithms}

\section{Applications of word vector}

\newpage
\subsubsection{Word2Vec}

BERT conditions on the rest of the input sentence.

BERT uses words, subwords and individual characters (in total 30'000) that are then used as tokens.

Idea is to do the following:
Concepts (and thus words), are represented across multiple contexts.
We can create probabilistic word-embeddings by sampling from a corpus the context of a specific word.
From multiple samples of the context-embedding-vector, we take the mean and stddev, or create a GMM if there are multimodal logic (we can check this multimodality by runniing some sort of rejection sampling algorithm).
Then we have a probability density function (one for each language), and can map one into another.

Perhaps we could split up too high-variance embeddings to multimodal embeddings, depending on their use-cases.

This allows for interpretability in polysemy sentences.

Not using more complex flows implies that the flow itself is not the bottleneck (they probably tried out other flows as well).

Are the individual word-embeddings going to be one big blob with high variance, or is it going to be a multi-modal distribution...?

Another task we may look at is, from a given word-embedding, sample it's context. 
Not entirely sure how to do this with BERT and co.

At what point do we overfit, and at what point do we generalize?


\textbf{Artetxe bilingual token matching through unsupervised machine translation}

- Input is cross-lignual word-embeddings
- Build an unsupervised phrase-based statistical machine translation system
- Derive phrase-based embeddings from input-word-embeddings by taking top 400'000 bigrams and 400'000 trigrams -> take arithmetic mean of word-embedding
- score top 100 close phrases using softmax cosine similarity
- generation of synthetic parallel corpus using this approach
- Then use FastAlign to use parallel corpus to align words
- 

\chapter{Analysis of the current state of the art}
- Hubness problem (MUSE, and CSLS)

Polysemy:
a word may have multiple meanings.
in a cross-lingual embedding space, this feeling is amplified.
there's some work in multi-sense embedding.
this should enable to capture more fine-grained embeddings embeddings.

Upadhyay et al. argue that the choice of data is more important than the actual algorithm.

Definitely also look into this, [Analyzing the Limitations of Cross-lingual Word Embedding Mappings] seems to be an analysis of the difficulties etc. 

\section{On the Linear Separability of meaning within sampled BERT vectors}

\subsection{Motivation}

To see if there is any structure within BERT vectors w.r.t. the different meaning of one word, we ask ourselves whether or not different part of the meaning are at different locations of the embedding space produced by BERT.

\subsection{Experiment setup}

Let us regard a word $w$ in sentence $s$ is indexed by $i$.
When passed through the BERT model, the word $w$ produces an embedding $x_w$.
When we sample BERT embeddings for a single word $w$ for $n=500$ sentences. 

Let us restrict the choice of words $w$ on polysemous and ambigious words which carry multiple meanings. 
Specifically, $w$ is chosen in such a way that each meaning has more than 30 samples in each class when chosen from within the SemCor dataset. 

\begin{center}

% Words used to test BERT-sampled vectors for linear separability of lexical features
\begin{tabular}{ | c | c | c | }
\hline
\multicolumn{3}{ | c | }{Words used to test BERT-sampled vectors for linear separability of lexical features} \\
\hline
 was & is & be \\ 
 \hline
 are & more & one \\  
 \hline
 first & only & time \\
 \hline
\end{tabular}
\end{center}

When we sample 
Specifically, the experiment setup looks as follows.

\begin{algorithm}[H]
\SetAlgoLined
\SetKwInOut{Input}{Input}
\Input{A target word $w_{\text{target}}$; The latent dimensionality $k$for PCA;}
\KwResult{Accuracy of a logistic regression classifier}
 $\mathbf{D}, \mathbf{y} \leftarrow $  sample up to 500 sentences (as much as available) from the SemCor corpus which include the word $w_{\text{target}}$, along with the corresponding WordNet meaning-id\;

$ \mathbf{X} \leftarrow BERT( \mathbf{D} )$ i.e. pass each sentence through BERT and retrieve the resulting word embedding $x_w$ as defined in the above section\;
 
$ \mathbf{X} \leftarrow oversample( \mathbf{X}, \mathbf{y} )$ such that we don't have dominating classes\;
 
$ \mathbf{X_\text{train}}, \mathbf{X_\text{test}}, \mathbf{y_\text{train}}, \mathbf{y_\text{test}} \leftarrow trainTestSplit( \mathbf{X}, \mathbf{y}, testProportion=0.4 )$ \;

$ \mathbf{X_\text{train}} \leftarrow StandardScaler( \mathbf{X_\text{train}})$ such that all the data is normalized\;

$ \mathbf{X_\text{train}} \leftarrow PCA( \mathbf{X_\text{train}}, k )$ such that all the data is normalized\;

$ model \leftarrow LogisticRegression( \mathbf{X_\text{train}}, \mathbf{y_\text{train}} )$ \;
    
$ \mathbf{\hat{y}_\text{test}} \leftarrow model.predict(\mathbf{X_\text{test}})$ \;

$ accuracy, confusionMatrix \leftarrow loss(\mathbf{y_\text{test}}, \mathbf{\hat{y}_\text{test}}) $ \;
    
return $ accuracy, confusionMatrix $\;
    
 \caption{Checks sampled BERT vectors for linear interpretability by meaning}
\end{algorithm}

We use a binary 
%TODO what is the formal name of the loss function that we are using..?

First of all, we sample $n=500$ sentences from the news corpus for a target word $\hat{w}$, which has index $i$ in sentence $s$.

Each sentence, we pass through the 

The BERT model takes a sequence of size (up to) 512 elements as input.

We sample $n=500$ vectors from BERT.
This is the vector at the output embedding layer of BERT, which has the same location as the input to the BERT model.

%TODO insert graphic of which embedding is used..


\subsection{Results}

We run the above experiment for a set of different $k$ to build up an intuition of how well different dimensionalites still capture the meaning in different locations of the vector space. 


\section{On the Clusterability of meaning within sampled BERT vectors}

\subsection{Motivation}
\subsection{Experiment setup}

%TODO include optimization loop over all models

\begin{algorithm}[H]
\SetAlgoLined
\SetKwInOut{Input}{Input}
\Input{A target word $w_{\text{target}}$; The latent dimensionality $k$for PCA;}
\KwResult{The associated cluster-id with each sampled sentence}
 $\mathbf{D}, \mathbf{y} \leftarrow $  sample up to 500 sentences from the SemCor corpus which include the word $w_{\text{target}}$, along with the corresponding WordNet meaning-id, and compensate more sentences by sampling from the news corpus if less than 500 sentences are available in the SemCor sentence\;

$ \mathbf{X} \leftarrow BERT( \mathbf{D} )$ i.e. pass each sentence through BERT and retrieve the resulting word embedding $x_w$ as defined in the above section\;
 
$ \mathbf{X} \leftarrow oversample( \mathbf{X}, \mathbf{y} )$ such that we don't have dominating classes\;
 
$ \mathbf{X_\text{train}}, \mathbf{X_\text{test}}, \mathbf{y_\text{train}}, \mathbf{y_\text{test}} \leftarrow trainTestSplit( \mathbf{X}, \mathbf{y}, testProportion=0.4 )$ \;

$ \mathbf{X_\text{train}} \leftarrow StandardScaler( \mathbf{X_\text{train}})$ such that all the data is normalized\;

$ \mathbf{X_\text{train}} \leftarrow PCA( \mathbf{X_\text{train}}, k )$ such that all the data is normalized\;

$ model \leftarrow LogisticRegression( \mathbf{X_\text{train}}, \mathbf{y_\text{train}} )$ \;
    
$ \mathbf{\hat{y}_\text{test}} \leftarrow model.predict(\mathbf{X_\text{test}})$ \;

$ accuracy, confusionMatrix \leftarrow loss(\mathbf{y_\text{test}}, \mathbf{\hat{y}_\text{test}}) $ \;
    
return $ accuracy, confusionMatrix $\;
    
 \caption{Checks sampled BERT vectors for clusters by  meaning}
\end{algorithm}

\subsection{Results}

\section{Correlation between Part of Speech and Context within BERT}

\subsection{Motivation}
\subsection{Experiment setup}
\subsection{Results}

It is apparent that there is a strong relation between PoS and meaning. 
Especially "erstarrte" Verben are a strong part of this

%\chapter{Evaluation} 
\chapter{Our Method}

\subsection{BERnie PoS}
\subsubsection{Motivation}
\subsubsection{Experiment setup}

\subsection{BERnie Meaning}
\subsubsection{Motivation}
\subsubsection{Experiment setup}

\subsection{BERnie Meaning with additional pre-training}
\subsubsection{Motivation}
\subsubsection{Experiment setup}

\subsection{Compressing the non-lexical out}
\subsubsection{Motivation}
\subsubsection{Experiment setup}

\chapter{Further Work}

Can be used for more unstructured data, like graphs.

\
%\chapter{Summary and Conclusions} 
\chapter{Evaluation}

\chapter{Conclusion}

\appendix
\singlespacing

\bibliographystyle{unsrt} 
\bibliography{dissertation} 

\end{document}
